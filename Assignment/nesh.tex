\documentclass{article}
\usepackage{amsmath}
\usepackage{geometry}
\usepackage{amssymb}
\usepackage{gvv}
\usepackage{gvv-book}
\usepackage{graphicx}
\usepackage{enumitem}
\usepackage{float}


\begin{document}

\section{ALGEBRA}
\begin{enumerate}
\item The function $f(x) = x|x|$ is
\begin{enumerate}
\item continuous and differentiable at $x = 0.$
\item continuous but not differentiable at $x = 0.$
\item differentiable but not continuous at $x = 0.$
\item neither differentiable nor continuous at $x = 0.$
\end{enumerate}
\item The objective function $Z = ax + by$ of an LPP has maximum value 42 at $(4, 6)$ and minimum value 19 at $(3, 2)$. Which of the following is true?
\begin{enumerate}
\item $a = 9, b = 1$
\item $a = 5, b = 2$
\item $a = 3, b = 5$
\item $a = 5, b = 3$
\end{enumerate}
\item The corner points of the feasible region of a linear programming problem are $(0, 4)$, $(8, 0)$, and $(\frac{20}{3}, \frac{4}{3})$. If $Z = 30x + 24y$ is the objective function, then (maximum value of $Z$ - minimum value of $Z$) is equal to
\begin{enumerate}
\item 40
\item 96
\item 120
\item 136
\end{enumerate}
\item Solve the following linear programming problem graphically:\\ Maximize :  $Z = x + 2y$ \\
	subject to constraints:\\ 
		 \centerline{$x + 2y \geq 100,$}\\
		 \centerline{$2x - y \leq 0,$}\\
		 \centerline{$2x + y \leq 200,$}\\
		 \centerline{$x \geq 0,y \geq 0.$}
\item A function $f: [-4, 4] \to [0, 4]$ is given by $f(x) = \sqrt{16-x^2}$. Show that $f$ is an onto function but not a one-one function. Further, find all possible values of $a$ for which $f(a) = \sqrt{7}$.
\item The use of electric vehicles will curb air pollution in the long run.\\
\begin{figure}[H]
\centering
	\includegraphics[width=\columnwidth]{figs/Thrinesh 3.PNG}
	\caption{electric charging point}
\label{fig}
\end{figure}
The use of electric vehicles is increasing every year and estimated electric vehicles in use at any time \(t\) is given by the function \(V\):\\
		\begin{align}
		V(t)=\frac{1}{5} t^3-\frac{5}{2}t^2+25t-2
		\end{align}
where \(t\) represents the time and t=1, 2, 3....corresponds to year 2001, 2002, 2003, ...respectively.\\
Based on the above information, answer the following questions :
\begin{itemize}
    \item [(i)] Can the above function be used to estimate number of vechicles in the year 2000 ? Justify.
    \item [(ii)] Prove that the function \(V(t)\) is an increasing function.
    \end{itemize}
\end{enumerate}

\section{DIFFERENTIATION}
\begin{enumerate}
\item If $\frac{d}{dx}f(x) = 2x +\frac{3}{x}$ and $f(1) = 1$, then $f(x)$ is
\begin{enumerate}
\item $x^2 + 3 \log |x| + 1$
\item $x^2 + 3 \log |x|$
\item $2 -\frac{3}{x^2}$
\item $x^2 + 3 \log |x| - 4$
\end{enumerate}

\item Degree of the differential equation $\sin x + \cos (\frac{dy}{dx}) = y^2$ is
\begin{enumerate}
\item 2
\item 1
\item not defined
\item 0
\end{enumerate}

\item The integrating factor of the differential equation 
	\begin{align}
		(1 - y^2) \frac{dx}{dy} + yx = ay, (-1 < y < 1) is
	\end{align}
\begin{enumerate}
\item $\frac{1}{y^2-1}$
\item $\frac{1}{\sqrt{y^2 - 1}}$
\item $\frac{1}{1 - y^2}$
\item $\frac{1}{\sqrt{1-y^2}}$
\end{enumerate}
\item Anti-derivative of $\frac{\tan x-1}{\tan     x+1}$ with respect to $x$ is: 
\begin{enumerate}           
\item $\sec^2(\frac{\pi}{4}-x)+c$ 
\item $-\sec^2(\frac{\pi}{4}-x)+c$  
\item $\log\left|\sec(\frac{\pi}{4}-x)\right|+c$  
\item $-\log\left|\sec(\frac{\pi}{4}-x)\right|+c$

\end{enumerate}
\item If $\tan\left(\frac{x+y}{x-y}\right) = k$, then $\frac{dy}{dx}$ is equal to
\begin{enumerate}
\item $\frac{-y}{x}$
\item $\frac{y}{x}$
\item $\sec^2(\frac{y}{x})$
\item $-\sec^2(\frac{y}{x})$
\end{enumerate}

\item If $y = \sqrt{ax+b}$, prove that $y \left(\frac{d^2y}{dx^2}\right) + \left(\frac{dy}{dx}\right)^2 = 0$.

\item If $f(x) =
\begin{cases}
ax + b  ; & 0 < x \leq 1 \\
2x^2 - x ; & 1 < x < 2
\end{cases}$ is a differentiable function in $(0, 2)$, then find the values of $a$ and $b$.
\item Find the general solution of the differential equation: \begin{align}
		(xy - x^2) \, dy = y^2 \, dx.
\end{align}

\item  Find the general solution of the differential equation:
	\begin{align}
		(x^2 + 1)\frac{dy}{dx} + 2xy = \sqrt{x^2+4}.
	\end{align}
	
\end{enumerate}

\section{GEOMETRY}
\begin{enumerate}
\item Equation of line passing through origin and making $30^\circ$, $60^\circ$ and $90^\circ$ with $x$, $y$, $z$ axes respectively is
\begin{enumerate}
\item $\frac {2x}{\sqrt{3}} = \frac{y}{2} =\frac {z}{0}$
\item $\frac {2x}{\sqrt{3}} =\frac {2y}{1} =\frac {z}{0}$
\item $2x = \frac{2y}{\sqrt{3}} = \frac {z}{1}$
\item $\frac{2x}{\sqrt{3}} = \frac{2y}{1}= \frac{z}{1}$
\end{enumerate}
\item If the equation of a line is $x = ay + b$, $z = cy + d$, then find the direction ratios of the line and a point on the line.
\item If the circumference of a circle is increasing at the constant rate, prove that rate of change of  area of circle is directly proportional to its radius.
\item Find the equations of the diagonals of the parallelogram $PQRS$ whose vertices are $P(4, 2, -6)$, $Q(5, -3, 1)$, $R(12, 4, 5)$ and $S(11, 9, -2)$. Use these equations to find the point of intersection of diagonals.
\item A line $l$ passes through point $(-1, 3, -2)$ and is perpendicular to both the lines $\frac{x}{1} = \frac{y}{2} = \frac{z}{3}$ and $\frac{x + 2}{-3} = \frac{y-1}{2} = \frac{z + 1}{5}$. Find the vector equation of the line $l$. Hence, obtain its distance from origin.
\item Engine dispalcement is the measure of the cylinder volume swept by all the pistons of a piston engine. The piston moves inside the cylinder bore
\begin{figure}[H]
    \centering
	\includegraphics[width=\columnwidth]{figs/Thrinesh 1.PNG}
        \caption{Cylinder bore}
     \label{fig:fig-1}
\end{figure}
The cylinder bore in the form of circular cylinder open at the top is to be made from a metal sheet of area $75\pi cm^2$.\\
Based on the above information, answer the following questions :
\begin{enumerate}[label=(\roman*)]
\item If the radius of cylinder is $r$ cm and height is $h$ cm, then write the volume V of cylinder in terms of radius $r$.
\item  Find $\frac{dV}{dr}$.
\item  Find the radius of the cylinder when its volume is maximum.
\item For maximum volume, $h>r$. State true or false and justify.
\end{enumerate}
\end{enumerate}

\section{INTEGRATION}
\begin{enumerate}
    \item  Evaluate  $\int_{\log \sqrt{2}}^{\log \sqrt{3}} \frac{1}{(e^x + e^{-x})(e^x - e^{-x})}dx$
    \item  Evaluate $\int_{-1}^{1} |x^4 - x| \, dx$.
    \item  Find $\int \frac{\sin^{-1} x}{(1 - x^2)^{3/2}} \, dx$.
    \item Find $\int e^x(\frac{ 1 - \sin x}{1 - \cos x}) \, dx$
    \item  Using Integration, find the area of the triangle whose vertices are $(-1, 1)$, $(0, 5)$ and $(3, 2)$.
\end{enumerate}

\section{MATRIX}
\begin{enumerate}
\item  If $(a, b)$, $(c, d)$, and $(e, f)$ are the vertices of $\triangle ABC$ and $\Delta$ denotes the area of $\triangle ABC$, then $\mydet
{ a & c & e \\
 b & d & f &\\
 1 & 1 & 1}^2 $ is equal to
\begin{enumerate}
\item  $2\Delta^2$
\item $4\Delta^2$
\item $2\Delta$
\item $4\Delta$
\end{enumerate}
\item If $A$ is a $2\times 3$ matrix such that $AB$ and $AB'$ both are defined, then order of the matrix $B$ is
	\begin{enumerate}
		\item $2\times2$
		\item $2\times1$
		\item $3\times2$
		\item $3\times3$
	\end{enumerate}
\item If $\myvec
	{2 & 0\\
        5 & 4}$ = P + Q,  where $P$ is a symmetric and $Q$ is a skew symmetric matrix, then $Q$ is equal to
		\begin{enumerate}
			\item $\myvec
				{2 & \frac{5}{2}\\
				\frac{5}{2} & 4}$
			\item $\myvec
				{0 & \frac{-5}{2}\\
				\frac{5}{2} & 0}$
			\item $\myvec
				{0 & \frac{5}{2}\\
				\frac{-5}{2} & 0}$
			\item $\myvec
				{2 & \frac{-5}{2}\\
				\frac{5}{2} & 4}$
		\end{enumerate}
\item  If$\myvec
	       {1 & 2 & 1\\
		2 & 3 & 1\\
		3 & a & 1}$  is a non-singular matrix and $a \in A$, then the set $A$ is
		\begin{enumerate}
			\item $\mathbb{R}$
	\item $\{0\}$
			\item $\{4\}$
			\item $\mathbb{R}-\{4\}$
		\end{enumerate}
\item If $\mydet {A} $= $\mydet{kA}$, where $A$ is a square matrix of order 2, then sum of all possible values of $k$ is
	\begin{enumerate}
		\item $1$
		\item $-1$
		\item $2$
		\item $0$
	\end{enumerate}
\item If $A$ = $\myvec
	        {-3 & -2 & -4 \\
		2 & 1 & 2 \\
                2 & 1 & 3}$ and $B$ = $\myvec		               {1 & 2 & 0 \\ 
                -2 & -1 & -2 \\ 
                0 & -1 & 1}$, then find $AB$ and use it to solve the following system of equations:  
\[
\begin{aligned}
x - 2y &= 3 \\
2x - y - z &= 2 \\
-2y + z &= 3
\end{aligned}
\]		
\item If $f(\alpha) = \myvec
	{\cos \alpha & -\sin \alpha &  0 \\
	\sin \alpha & \cos \alpha & 0 \\ 
	0 & 0 & 1}$, then prove that $f(\alpha) \cdot f(-\beta)$ = $f(\alpha - \beta)$
\end{enumerate}

\section{PROBABILITY}
\begin{enumerate}
\item If $A$ and $B$ are two events such that $P(A / B) = 2 \times P(B / A)$ and $P(A) + P(B) = \frac{2}{3}$, then $P(B)$ is equal to
\begin{enumerate}
\item $\frac{2}{9}$
\item $\frac{7}{9}$
\item $\frac{4}{9}$
\item $\frac{5}{9}$
\end{enumerate}
\item Two balls are drawn at random one by one with replacement from an urn containing equal numbers of red balls and green balls. Find the probability distribution of  number of red balls. Also, find the mean of the random variable.
\item A and B throw a die alternately till one of them gets a $6$ and wins the game. Find their respective probabilities of winning, if A starts the game first.
\item Recent studies suggest that roughly 12\% of the world population is left handed \\
\begin{figure}[H]
\centering
	\includegraphics[width=\columnwidth]{figs/Thrinesh 2.PNG}
\label{fig}
\end{figure}
Depending upon the parents, the chances of having a left handed child are as folws;
\begin{enumerate}
\item When both father and mother are left handed:\\
Chances of left handed child is 24\%.
\item When father is right handed and mother is left handed:\\
Chances of left handrd child is 22\%.
\item  When father is left handed and mother is right handed:\\
Chances of left handed child is 17\%.
\item  When both faher and mother are right handed :\\
Chances of left handed child is 9\%.
\end{enumerate}
Assuming that $P(A) = P(B) = P(C)= P(D) =\frac{1}{4}$ and $L$ denotes the event that child is left handed\\
Based on the above information, answer the following questions:
		\begin{enumerate}[label=(\roman*)]
			\item Find $P(L/C)$
			\item Find $P(\overline{L}/A)$ 
			\item Find $P(A/L)$
                        \item Find the probability that a randomly selected child is left handed given that exactly one of the parents is left handed.
		\end{enumerate}
\end{enumerate}

\section{TRIGONOMETRY}
\begin{enumerate}
\item Evaluate $\sin^{-1}(\sin\frac{3\pi}{4}) + \cos^{-1}(\cos \pi) + \tan^{-1}(1)$.	

\item Draw the graph of $\cos^{-1} x$, where $x \in [-1, 0]$. Also, write its range.	
\item \textbf{Assertion (A) :} If a line makes angles $\alpha$, $\beta$, $\gamma$ with the positive direction of the coordinate axes, then $\sin^2 \alpha + \sin^2 \beta + \sin^2 \gamma = 2$. \\
	\textbf{Reason (R) :} The sum of squares of the direction cosines of a line is $1$.
\item \textbf{Assertion (A) :} Maximum value of $(\cos^{-1} x)^2$ is $\pi ^2$.\\
	\textbf{Reason (R) :} Range of the principal value branch of $\cos ^{-1} x$  is  $[\frac{-\pi}{2},\frac{\pi}{2}]$.

\end{enumerate}

\section{VECTORS}
\begin{enumerate}
\item Unit vector along $\overrightarrow{PQ}$, where coordinates of $P$ and $Q$ respectively are $(2, 1, -1)$ and $(4, 4, -7)$, is
\begin{enumerate}
\item ${2\hat{i} + 3\hat{j} - 6\hat{k}}$
\item ${-2\hat{i} - 3\hat{j} + 6\hat{k}}$
\item $\frac{-2\hat{i}}{7} -\frac{3\hat{j}}{7} +\frac{6\hat{k}}{7}$
\item $\frac{2\hat{i}}{7} + \frac{3\hat{j}}{7} - \frac{6\hat{k}}{7}$
\end{enumerate}



\item If $|\overrightarrow{a} \times \overrightarrow{b}| = \sqrt{3}$ and $\overrightarrow{a} \cdot \overrightarrow{b} = -3$, then angle between $\overrightarrow{a}$ and $\overrightarrow{b}$ is
\begin{enumerate}
\item $\frac{2\pi}{3}$
\item $\frac{\pi}{6}$
\item $\frac{\pi}{3}$
\item $\frac{5\pi}{6}$
\end{enumerate}

\item If in $\triangle ABC$, $\overrightarrow{BA}$ = $2\overrightarrow{a}$ and $\overrightarrow{BC}$ =$ 3\overrightarrow{b}$, then $\overrightarrow{AC}$ is
\begin{enumerate}
\item $2\overrightarrow{a} + 3\overrightarrow{b}$
\item $2\overrightarrow{a} - 3\overrightarrow{b}$
\item $3\overrightarrow{b} - 2\overrightarrow{a}$
\item $-2\overrightarrow{a} - 3\overrightarrow{b}$
\end{enumerate}

\item  If $\overrightarrow{a}$, $\overrightarrow{b}$, $\overrightarrow{c}$ are three non-zero unequal vectors such that $\overrightarrow{a} \cdot \overrightarrow{b} = \overrightarrow{a} \cdot \overrightarrow{c}$, then find the angle between $\overrightarrow{a}$ and $\overrightarrow{b}-\overrightarrow{c}$.
\end{enumerate}

\end{document}
